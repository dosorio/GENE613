\documentclass[12pt,a4paper]{paper}
\usepackage[utf8]{inputenc}
\usepackage[spanish]{babel}
\usepackage{amsmath}
\usepackage{tikz}
\usepackage{pgfplots}
\usepackage[makeroom]{cancel}
\usepackage{enumitem}
\usepackage{amsfonts}
\usepackage{amssymb}
\usepackage[left=2cm,right=2cm,top=2cm,bottom=2cm]{geometry}
\usepackage{Sweave}
\begin{document}
\title{GENE613 - Homework 4\\\small{Daniel Osorio - dcosorioh@tamu.edu\\Department of Veterinary Integrative Biosciences\\Texas A\&M University}}
\maketitle
\Sconcordance{concordance:HW4_DanielOsorio.tex:HW4_DanielOsorio.Rnw:%
1 11 1 1 0 20 1 1 4 6 0 1 2 8 1 1 2 7 0 2 2 7 0 2 2 7 0 1 2 4 1 1 9 8 0 %
1 1 7 0 1 2 1 8 7 0 1 1 6 0 2 2 7 0 2 2 7 0 2 2 7 0 1 2 1 6 5 0 1 1 6 0 %
1 2 1 4 3 0 1 1 6 0 1 2 21 1}

Decay of disequilibrium is a function of $r$ and generations $t$
\begin{equation*}
D^{t}_{AB} = \left(1-r\right)^{t}D_{AB}^{0}
\end{equation*}
\begin{enumerate}
\item As $t \rightarrow \infty$ what happens to $D^{t}_{AB}$?\\
As $D^{t}_{AB}$ changes in function of  $t$ and $(1-r)^{t}$ will be always a smaller number as $t$ increases, the $\left(1-r\right)^{t}D_{AB}^{0}$ will tend to 0.
\item Derive a general solution for the number of generations $t$ required to move from initial disequilibrium $D_{AB}^{0}$ to a target, or eventual disequilibrium $D_{AB}^{t}$
\begin{equation}
\begin{split}
D^{t}_{AB} &= \left(1-r\right)^{t} \times D_{AB}^{0}\\
log\left(D^{t}_{AB}\right) &= t \times log\left(1-r\right) + log\left(D_{AB}^{0}\right)\\
log\left(D^{t}_{AB}\right) - log\left(D_{AB}^{0}\right) &= t \times log\left(1-r\right)\\
log\left(D^{t}_{AB}\right) - log\left(D_{AB}^{0}\right) &= t \times log\left(1-r\right)\\
\frac{log\left(D^{t}_{AB}\right) - log\left(D_{AB}^{0}\right)}{log\left(1-r\right)} &= t
\end{split}
\end{equation}
\begin{Schunk}
\begin{Sinput}
> requiredGenerations <- function(start, end, r) {
+   return (ceiling((log(end) - log(start)) / log(1 - r)))
+ }
\end{Sinput}
\end{Schunk}
\item If initial disequilibrium is 0.2 and recombination rate between a pair of loci is 0.2
\begin{enumerate}
\item Evaluate the magnitude of this disequilibrium\\
As the disequilibrium coefficient $D_{AB}$ varies in magnitude between a minimum of -0.25 and a maximum of +0.25 when there are only repulsion gametes or they are not present, a $D_{AB}$ = 0.2 is a high value that represents the 80\% of the maximum disequilibrium possible.
\item Interpret and evaluate the magnitude of this recombination rate\\
As the recombination rate ($r$) ranges in value between 0  and 0.5 the maximum is at 0.5 because, with an independent assortment of the two loci, one-half of the gametes produced will still be the parental type. A $r$ value of 0.2 is the 40\% of the maximum recombination rate possible.
\item How many generations would be required to reach a disequilibrium value of 
\begin{enumerate}
\item 0.1? 
\begin{Schunk}
\begin{Sinput}
> requiredGenerations(start = 0.2, r = 0.2, end = 0.1)
\end{Sinput}
\begin{Soutput}
[1] 4
\end{Soutput}
\end{Schunk}
\item 0.05? 
\begin{Schunk}
\begin{Sinput}
> requiredGenerations(start = 0.2, r = 0.2, end = 0.05)
\end{Sinput}
\begin{Soutput}
[1] 7
\end{Soutput}
\end{Schunk}
\item 0?
\begin{Schunk}
\begin{Sinput}
> requiredGenerations(start = 0.2, r = 0.2, end = 0)
\end{Sinput}
\begin{Soutput}
[1] Inf
\end{Soutput}
\end{Schunk}
\end{enumerate}
\end{enumerate}
\item Given $P_{AB} = 0.6$, $P_{Ab} = 0.1$, $P_{aB} = 0.2$ and $P_{ab} = 0.1$ calculate:
\begin{enumerate}
\item Allele frquencies
\begin{Schunk}
\begin{Sinput}
> alleleFrequencies <- function(AB, Ab, aB, ab) {
+   return(c(
+     A = AB + Ab,
+     B = AB + aB,
+     a = aB + ab,
+     b = Ab + ab
+   ))
+ }
> alleleFrequencies(AB = 0.6, Ab = 0.1, aB = 0.2, ab = 0.1)
\end{Sinput}
\begin{Soutput}
  A   B   a   b 
0.7 0.8 0.3 0.2 
\end{Soutput}
\end{Schunk}
\item $D_{AB}$
\begin{Schunk}
\begin{Sinput}
> D <- function(a = NULL, A=NULL, b=NULL, B=NULL, observedF) {
+   if(length(c(A,B)) > 1 | length(c(a,b)) > 1){
+     return(observedF - c((A*B),(a*b)))
+   } else{
+     return(c((A*b),(a*B)) - observedF)
+   }
+ }
> D(A = 0.7, B = 0.8, observedF = 0.6)
\end{Sinput}
\begin{Soutput}
[1] 0.04
\end{Soutput}
\end{Schunk}
\item $D_{Ab}$
\begin{Schunk}
\begin{Sinput}
> D(A = 0.7, b = 0.2, observedF = 0.1)
\end{Sinput}
\begin{Soutput}
[1] 0.04
\end{Soutput}
\end{Schunk}
\item $D_{aB}$
\begin{Schunk}
\begin{Sinput}
> D(a = 0.3, B = 0.8, observedF = 0.2)
\end{Sinput}
\begin{Soutput}
[1] 0.04
\end{Soutput}
\end{Schunk}
\item $D_{ab}$
\begin{Schunk}
\begin{Sinput}
> D(a = 0.3, b = 0.2, observedF = 0.1)
\end{Sinput}
\begin{Soutput}
[1] 0.04
\end{Soutput}
\end{Schunk}
\item $D'$
\begin{Schunk}
\begin{Sinput}
> lewontinD <- function (DAB, PA, PB, Pa, Pb) {
+   ifelse(test = DAB > 0,
+          yes = (DAB / min(c(PA * Pb, Pa * PB))),
+          no = (DAB / min(c(PA * PB, Pa * Pb))))
+ }
> lewontinD(DAB = 0.04, PA = 0.7, PB = 0.8, Pa = 0.3, Pb = 0.2)
\end{Sinput}
\begin{Soutput}
[1] 0.2857143
\end{Soutput}
\end{Schunk}
\item $r^{2}$
\begin{Schunk}
\begin{Sinput}
> rSquared <- function(DAB, PA, PB, Pa, Pb) {
+   (DAB ^ 2) / (PA * Pa * PB * Pb)
+ }
> rSquared(DAB = 0.04, PA = 0.7, PB = 0.8, Pa = 0.3, Pb = 0.2)
\end{Sinput}
\begin{Soutput}
[1] 0.04761905
\end{Soutput}
\end{Schunk}
\end{enumerate}
\item Show thar $D_{ab}$ = $D_{AB}$
\begin{equation}
\begin{split}
D_{ab} &= D_{AB}\\
P(ab) - P(a) \times P(b) &= P(AB) - (P(A) \times P(B))\\
P(ab) - \left[(1-P(A)) \times (1-P(B))\right] &= P(AB) - (P(A) \times P(B))\\
P(ab) - \left[1 - P(A) -P(B) + (P(A) \times P(B))\right] &= P(AB) - (P(A) \times P(B))\\
P(ab) - 1 + P(A) + P(B) \bcancel{- (P(A) \times P(B))} &= P(AB) \bcancel{- (P(A) \times P(B))}\\
P(ab) - 1 + P(A) + P(B)  &= P(AB)\\
\bcancel{P(ab)} - P(AB) - P(Ab) - P(aB) \bcancel{- P(ab)} + P(A) + P(B)  &= P(AB)\\
- P(AB) - P(Ab) - P(aB) + P(A) + P(B)  &= P(AB)\\
- P(AB) - P(Ab) - P(aB) + P(AB) + P(Ab) + P(B)  &= P(AB)\\
\bcancel{- P(AB)} - P(Ab) - P(aB) \bcancel{+ P(AB)} + P(Ab) + P(B)  &= P(AB)\\
\bcancel{- P(Ab)} - P(aB) \bcancel{+ P(Ab)} + P(B)  &= P(AB)\\
- P(aB) + P(B)  &= P(AB)\\
\bcancel{- P(aB)} + P(AB) \bcancel{+ P(aB)} &= P(AB)\\
P(AB) &= P(AB)\\
\end{split}
\end{equation}
\end{enumerate}
\end{document}
