\documentclass[12pt,a4paper]{paper}
\usepackage[utf8]{inputenc}
\usepackage[spanish]{babel}
\usepackage{amsmath}
\usepackage{tikz}
\usepackage{pgfplots}
\usepackage[makeroom]{cancel}
\usepackage{enumitem}
\usepackage{amsfonts}
\usepackage{amssymb}
\usepackage[left=2cm,right=2cm,top=2cm,bottom=2cm]{geometry}
\usepackage{Sweave}
\begin{document}
\title{GENE613 - Homework 6\\\small{Daniel Osorio - dcosorioh@tamu.edu\\Department of Veterinary Integrative Biosciences\\Texas A\&M University}}
\maketitle
\Sconcordance{concordance:HW6_DanielOsorio.tex:HW6_DanielOsorio.Rnw:%
1 11 1 1 0 3 1 1 9 1 1 1 2 13 0 1 2 1 30 29 0 1 3 17 0 1 2 4 1 1 23 22 %
0 1 3 17 0 2 2 1 0 1 1 1 4 2 0 1 2 3 1 6 0 1 1 7 0 2 2 1 0 1 13 12 0 2 %
1 6 0 1 1 7 0 2 2 1 0 1 13 12 0 2 1 6 0 1 1 7 0 1 2 1 1}

\begin{enumerate}
\item[] Use the following pedigree information for $F_{2}$ Nellore-Angus cross females to:
\begin{Schunk}
\begin{Sinput}
> nelloreAngus
\end{Sinput}
\begin{Soutput}
  Individual Phenotype Sire  Dam
1       189P        20 551G 911H
2       374R       -14 551G 911H
3       527S        11 551G 911H
4       602W        -1 494S 189P
5       686W        21 494S 527S
6       919Z       -19 494S 374R
\end{Soutput}
\end{Schunk}
\item Generate a matrix of additive relationship values $a_{xy}$
\begin{Schunk}
\begin{Sinput}
> additiveRelationship <- function(data, orderedIDs){
+   data <- as.matrix(data[, c("Individual", "Sire", "Dam")])
+   n <- length(orderedIDs)
+   aR <- matrix(
+     data = 0,
+     nrow = n,
+     ncol = n,
+     dimnames = list(orderedIDs, orderedIDs)
+   )
+   diag(aR) <- 1
+   for(i in seq_len(n)){
+     for(j in seq_len(n)){
+       if(j > i){
+         row <- orderedIDs[i]
+         column <- orderedIDs[j]
+         parents <- data[data[,"Individual"] %in% column,2:3]
+         if(all(is.na(parents))){next()}
+         value <- (0.5 * (aR[row,parents[1]] + aR[row,parents[2]]))
+         aR[row,column] <- aR[column,row] <-  value
+       } else {
+       row <- orderedIDs[i]
+       parents <- data[data[,"Individual"] %in% row,2:3]
+       if(all(is.na(parents))){next()} else {
+       aR[row,row] <- (1+ 0.5 * (aR[parents[1],parents[2]]))}
+       }
+     }
+   }
+   return(aR)
+ }
> additiveRelationship(data = nelloreAngus,
+                      orderedIDs = c("551G", "911H", "494S", "374R",
+                              "189P", "527S", "919Z", "602W", "686W"))
\end{Sinput}
\begin{Soutput}
     551G 911H 494S 374R 189P 527S  919Z  602W  686W
551G 1.00 0.00  0.0 0.50 0.50 0.50 0.250 0.250 0.250
911H 0.00 1.00  0.0 0.50 0.50 0.50 0.250 0.250 0.250
494S 0.00 0.00  1.0 0.00 0.00 0.00 0.500 0.500 0.500
374R 0.50 0.50  0.0 1.00 0.50 0.50 0.500 0.250 0.250
189P 0.50 0.50  0.0 0.50 1.00 0.50 0.250 0.500 0.250
527S 0.50 0.50  0.0 0.50 0.50 1.00 0.250 0.250 0.500
919Z 0.25 0.25  0.5 0.50 0.25 0.25 1.000 0.375 0.375
602W 0.25 0.25  0.5 0.25 0.50 0.25 0.375 1.000 0.375
686W 0.25 0.25  0.5 0.25 0.25 0.50 0.375 0.375 1.000
\end{Soutput}
\end{Schunk}
\item[] Non-additive relatedness: sharing of \underline{genotypes identical by descent.} This is dependent upon individuals having $> 2$ common ancestor that contribute to the relationship.
\begin{equation}
d_{xy} = \frac{(a_{\text{sire of X, sire of Y}} \times a_{\text{dam of X, dam of Y}} + a_{\text{sire of X, dam of Y}} \times a_{\text{sire of Y, dam of X}})}{4}
\end{equation}
\item Generate a matrix of dominance relationship values $d_{xy}$. What family structures resulted in values $> 0$. \textbf{Answer:} Offspring related to 551G and 911H.
\begin{Schunk}
\begin{Sinput}
> dominanceRelationship <- function(data, orderedIDs){
+   data <- as.matrix(data[, c("Individual", "Sire", "Dam")])
+   n <- length(orderedIDs)
+   aR <- additiveRelationship(data = data, orderedIDs = orderedIDs)
+   dR <- matrix(
+     data = 0,
+     nrow = n,
+     ncol = n,
+     dimnames = list(orderedIDs, orderedIDs)
+   )
+   for(i in seq_len(n)){
+     for(j in seq_len(n)){
+         iP <- data[data[,"Individual"] %in% orderedIDs[i],2:3]
+         if(all(is.na(iP))){next()}
+         jP <- data[data[,"Individual"] %in% orderedIDs[j],2:3]
+         if(all(is.na(jP))){next()}
+         dR[i,j] <- 0.25* ((aR[iP[1],jP[1]] * aR[iP[2],jP[2]]) +
+           (aR[iP[1],jP[2]] * aR[jP[1],iP[2]]))
+     }
+   }
+   return(dR)
+ }
> dominanceRelationship(data = nelloreAngus, 
+                       orderedIDs = c("551G", "911H", "494S", "374R",
+                              "189P", "527S", "919Z", "602W", "686W"))
\end{Sinput}
\begin{Soutput}
     551G 911H 494S 374R 189P 527S  919Z  602W  686W
551G    0    0    0 0.00 0.00 0.00 0.000 0.000 0.000
911H    0    0    0 0.00 0.00 0.00 0.000 0.000 0.000
494S    0    0    0 0.00 0.00 0.00 0.000 0.000 0.000
374R    0    0    0 0.25 0.25 0.25 0.000 0.000 0.000
189P    0    0    0 0.25 0.25 0.25 0.000 0.000 0.000
527S    0    0    0 0.25 0.25 0.25 0.000 0.000 0.000
919Z    0    0    0 0.00 0.00 0.00 0.250 0.125 0.125
602W    0    0    0 0.00 0.00 0.00 0.125 0.250 0.125
686W    0    0    0 0.00 0.00 0.00 0.125 0.125 0.250
\end{Soutput}
\end{Schunk}
\item If you distinctly (one at a time) predict breeding values for 919Z from single phenotypes of the individuals in the first column, which individuals' phenotypes result in the highest and lowest predicted values for 919Z? Which are the best and the worst predictions and why?\textbf{Answer:} The individual phenotype from which results the highest predictive breeding value for 919Z is itself and the lowest predictive breeding value result from using 686W because it shows the highest inverse phenotype. On another hand, the better and worse prediction result for 919Z is from itself and 189P or 527S respectively.
\begin{Schunk}
\begin{Sinput}
> centeredP <- nelloreAngus$Phenotype - mean(nelloreAngus$Phenotype)
> names(centeredP) <- nelloreAngus$Individual
> aR <- additiveRelationship(data = nelloreAngus,
+                      orderedIDs = c("551G", "911H", "494S", "374R",
+                              "189P", "527S", "919Z", "602W", "686W"))
> Y <- centeredP[c("602W", "686W", "919Z")]
> X <- centeredP[c("189P", "527S", "374R")]
> h2 <- cov(Y,X)/var(X)
> aR["919Z",nelloreAngus$Individual] * h2 * nelloreAngus$Phenotype
\end{Sinput}
\begin{Soutput}
       189P        374R        527S        602W        686W        919Z 
  3.7969925  -5.3157895   2.0883459  -0.2847744   5.9802632 -14.4285714 
\end{Soutput}
\begin{Sinput}
> aR["919Z",nelloreAngus$Individual] * sqrt(h2)
\end{Sinput}
\begin{Soutput}
     189P      374R      527S      602W      686W      919Z 
0.2178587 0.4357174 0.2178587 0.3267880 0.3267880 0.8714347 
\end{Soutput}
\end{Schunk}
\item Using breeding values based on phenotypes of progeny or progeny groups, which sire would be predicted to sire progeny with higher values of phenotype? How reliable are those predictors? What should sire breeding values average? \textbf{Answer:} The sire with the highest predicted values of phenotype to their progeny is 551G. The sire breeding values average and accuracy of the predictors are shown below.
\begin{Schunk}
\begin{Sinput}
> EBV <- ACC <- NULL
> for (sire in c("551G", "494S")){
+   progeny <- centeredP[nelloreAngus$Sire %in% sire]
+   if(length(progeny) > 2){
+     numeratorEBV <- (2 * length(progeny) * h2)
+     denominatorEBV <- 4 + ((length(progeny) -1) * h2)
+     numeratorACC <- (length(progeny) * h2)
+     EBV <- c(EBV, ((numeratorEBV/denominatorEBV)*mean(progeny)))
+     ACC <- c(ACC, sqrt(numeratorACC/denominatorEBV))
+   } else {
+     EBV <- c(EBV, 0.5 * h2 * progeny)
+     ACC <- c(ACC, 0.5 * sqrt(h2))
+   }
+ }
> names(EBV) <- names(ACC) <- c("551G", "494S")
> EBV
\end{Sinput}
\begin{Soutput}
     551G      494S 
 2.201635 -2.201635 
\end{Soutput}
\begin{Sinput}
> ACC
\end{Sinput}
\begin{Soutput}
     551G      494S 
0.6425002 0.6425002 
\end{Soutput}
\end{Schunk}
\item Using breeding values based on phenotypes of progeny or progeny groups, which dam would be predicted to have progeny with higher values of phenotype? Why do accuracies differ from the dams' breeding values in this example? \textbf{Answer:} The dam with the highest values of phenotype predicted to their progeny is 527S. Accuracies differ between dams due to the number of progeny available to do the calculations.
\begin{Schunk}
\begin{Sinput}
> EBV <- ACC <- NULL
> for (dam in c("911H", "189P", "527S", "374R")){
+   progeny <- centeredP[nelloreAngus$Dam %in% dam]
+   if(length(progeny) > 2){
+     numeratorEBV <- (2 * length(progeny) * h2)
+     denominatorEBV <- 4 + ((length(progeny) -1) * h2)
+     numeratorACC <- (length(progeny) * h2)
+     EBV <- c(EBV, ((numeratorEBV/denominatorEBV)*mean(progeny)))
+     ACC <- c(ACC, sqrt(numeratorACC/denominatorEBV))
+   } else {
+     EBV <- c(EBV, 0.5 * h2 * progeny)
+     ACC <- c(ACC, 0.5 * sqrt(h2))
+   }
+ }
> names(EBV) <- names(ACC) <- c("911H", "189P", "527S", "374R")
> EBV
\end{Sinput}
\begin{Soutput}
     911H      189P      527S      374R 
 2.201635 -1.518797  6.834586 -8.353383 
\end{Soutput}
\begin{Sinput}
> ACC
\end{Sinput}
\begin{Soutput}
     911H      189P      527S      374R 
0.6425002 0.4357174 0.4357174 0.4357174 
\end{Soutput}
\end{Schunk}
\end{enumerate}
\end{document}
