\documentclass[12pt,a4paper]{paper}
\usepackage[utf8]{inputenc}
\usepackage[spanish]{babel}
\usepackage{amsmath}
\usepackage{tikz}
\usepackage{pgfplots}
\usepackage[makeroom]{cancel}
\usepackage{enumitem}
\usepackage{amsfonts}
\usepackage{amssymb}
\usepackage[left=2cm,right=2cm,top=2cm,bottom=2cm]{geometry}
\usepackage{Sweave}
\begin{document}
\title{GENE613 - Homework 7\\\small{Daniel Osorio - dcosorioh@tamu.edu\\Department of Veterinary Integrative Biosciences\\Texas A\&M University}}
\maketitle
\Sconcordance{concordance:HW7_DanielOsorio.tex:HW7_DanielOsorio.Rnw:%
1 11 1 1 0 21 1 1 25 3 1 1 5 20 1 1 5 3 1 1 5 3 1 1 5 39 1 1 10 6 1 1 5 %
2 1}

\begin{enumerate}
\item If an animal has available its own performance $(X_{1}=50)$, the average performance of 8 paternal half-sibs $(X_{2}=65)$, the average performance of 10 progeny $(X_{3}=40)$ and the performance of its sire $(X_{4}=75)$, all this information can be used to calculate the animal's EBV using the selection index approach, solving for \[P\underline{b}=\underline{r}\] Assume the phenotypic and additice genetic variances are 2250 and 860, respectively.
\begin{enumerate}
\item Construct the $P$ matrix as the variances and covariances for the sources of information.
\begin{equation*}
P = 
\begin{bmatrix}
\sigma^{2}_{P} & 0.25\sigma^{2}_{A} & 0.5\sigma^{2}_{A} & 0.5\sigma^{2}_{A}\\
0.25\sigma^{2}_{A} & \sigma^{2}_{P} & 0.125\sigma^{2}_{A} & 0.5\sigma^{2}_{A} \\
0.5\sigma^{2}_{A} & 0.125\sigma^{2}_{A} & \sigma^{2}_{P} & 0.25\sigma^{2}_{A} \\
0.5\sigma^{2}_{A} & 0.5\sigma^{2}_{A} & 0.25\sigma^{2}_{A} & \sigma^{2}_{P}\sigma^{2}_{A}
\end{bmatrix} = \begin{bmatrix}
2250 & 215 & 430 & 430\\
215 & 2250 & 107.5 & 430\\
430 & 107.5 & 2250 & 215\\
430 & 430 & 215 & 2250
\end{bmatrix}
\end{equation*}
\item Construct the $\underline{r}$ vector as the covariates of the animal's true BV with the sources of information.
\begin{equation*}
\underline{r} = \begin{bmatrix}\sigma^{2}_{A} \\ 0.25\sigma^{2}_{A} \\ 0.5\sigma^{2}_{A} \\ 0.5\sigma^{2}_{A}\end{bmatrix}= \begin{bmatrix}860 \\ 215 \\ 430 \\ 430\end{bmatrix}
\end{equation*}
\item Invert the $P$ matrix
\begin{equation*}
P^{-1} = \begin{bmatrix}
0.000478 & -0.000027 & -0.000083 & -0.000078 \\
-0.000027 & 0.000463 & -0.000009 & -0.000083\\
-0.000083 & -0.000009 & 0.000463 & -0.000027\\
-0.000078 & -0.000083 & -0.000027 & 0.000478
\end{bmatrix}
\end{equation*}
\item Pre-multiply the $P^{-1}$ matrix to the $\underline{r}$ vector and show the resulting $\underline{\hat{b}}$ values.
\begin{equation*}
\underline{\hat{b}} = P^{-1} \times \underline{r}
\end{equation*}
\begin{equation*}
\underline{\hat{b}} = \begin{bmatrix}
0.000478 & -0.000027 & -0.000083 & -0.000078 \\
-0.000027 & 0.000463 & -0.000009 & -0.000083\\
-0.000083 & -0.000009 & 0.000463 & -0.000027\\
-0.000078 & -0.000083 & -0.000027 & 0.000478
\end{bmatrix} \times \begin{bmatrix}860 \\ 215 \\ 430 \\ 430\end{bmatrix} = \begin{bmatrix}0.336 \\ 0.037 \\ 0.115 \\ 0.109\end{bmatrix}
\end{equation*}
\item Calculate the EBV as $I =$ index on the animal.
\begin{equation*}
I = \begin{bmatrix}50 & 65 & 40 & 75\end{bmatrix} \times \begin{bmatrix}0.336 \\ 0.037 \\ 0.115 \\ 0.109\end{bmatrix} = 31.966
\end{equation*}
\item Calculate the ACC value associated with the EBV.
\begin{equation*}
ACC = \sqrt{\left(\frac{\underline{r}}{\sigma^{2}_{A}}\right)'\times \underline{\hat{b}}} = \sqrt{\begin{bmatrix}1 & 0.25 & 0.5 & 0.5\end{bmatrix} \times \begin{bmatrix}0.336 \\ 0.037 \\ 0.115 \\ 0.109\end{bmatrix}} = 0.676
\end{equation*}
\end{enumerate}
\item Estimate breeding values on the 6 individuals below using BLUP and the MME. Weights are already adjusted for sex differences. Assume $\sigma^{2}_{P} = 2500$ and $\sigma^{2}_{A} = 900$
\begin{center}\begin{tabular}{|c||c|c|c|c|}
\hline
ID&SIRE&DAM&GC&WEIGHT\\
\hline
\hline
1&0&0&1&930\\
\hline
2&0&0&1&880\\
\hline
3&1&0&2&965\\
\hline
4&1&2&2&945\\
\hline
5&3&0&3&970\\
\hline
6&4&0&3&950\\
\hline
\end{tabular}\end{center}
\begin{enumerate}
\item Construct the MME
% \begin{equation}
% X = \begin{bmatrix}1&1&0&0\\1&1&0&0\\1&0&1&0\\1&0&1&0\\1&0&0&1\\1&0&0&1\end{bmatrix} \hspace{0.5cm} Z = \begin{bmatrix}1&0&0&0&0&0\\0&1&0&0&0&0\\0&0&1&0&0&0\\0&0&0&1&0&0\\0&0&0&0&1&0\\0&0&0&0&0&1\end{bmatrix} \hspace{0.5cm} y = \begin{bmatrix}930\\880\\965\\945\\970\\950\end{bmatrix} \hspace{0.5cm} \lambda = \frac{2500-900}{900} = 1.78
% \end{equation}
\begin{equation*}
\begin{bmatrix}X'X & X'Z\\Z'X & Z'Z + A^{-1}\lambda\end{bmatrix} \times \begin{bmatrix}\underline{\hat{\beta}}\\\underline{\hat{u}}\end{bmatrix} = \begin{bmatrix}X'\underline{y}\\Z'\underline{y}\end{bmatrix}
\end{equation*}
\begin{equation*}
\begin{bmatrix}6&2&2&2&1&1&1&1&1&1\\
2&2&0&0&1&1&0&0&0&0\\
2&0&2&0&0&0&1&1&0&0\\
2&0&0&2&0&0&0&0&1&1\\
1&1&0&0&4.26&0.89&-1.19&-1.78&0&0\\
1&1&0&0&0.89&3.67&0&-1.78&0&0\\
1&0&1&0&-1.19&0&3.96&0&-1.19&0\\
1&0&1&0&-1.78&-1.78&0&5.15&0&-1.19\\
1&0&0&1&0&0&-1.19&0&3.37&0\\
1&0&0&1&0&0&0&-1.19&0&3.37\end{bmatrix} \times \begin{bmatrix}\mu\\\beta_{1}\\\beta_{2}\\\beta_{3}\\u_{1}\\u_{2}\\u_{3}\\u_{4}\\u_{5}\\u_{6}\end{bmatrix}= \begin{bmatrix}5740\\1810\\1910\\1920\\930\\880\\965\\945\\970\\950\end{bmatrix}
\end{equation*}
\item After deleting the equation for $\mu$, show the solutions vector $(\hat{\beta},\hat{u})$ containing contemporary group effects and estimated breeding values.
\begin{equation*}
\begin{bmatrix}\beta_{1}\\\beta_{2}\\\beta_{3}\\u_{1}\\u_{2}\\u_{3}\\u_{4}\\u_{5}\\u_{6}\end{bmatrix} = \begin{bmatrix}0.77&0.21&0.11&-0.28&-0.27&-0.16&-0.26&-0.09&-0.12\\0.21&0.85&0.17&-0.26&-0.17&-0.34&-0.35&-0.17&-0.18\\0.11&0.17&0.8&-0.13&-0.08&-0.17&-0.18&-0.3&-0.3\\-0.28&-0.26&-0.13&0.46&0.1&0.24&0.27&0.12&0.14\\-0.27&-0.17&-0.08&0.1&0.44&0.09&0.24&0.06&0.11\\-0.16&-0.34&-0.17&0.24&0.09&0.48&0.21&0.22&0.13\\-0.26&-0.35&-0.18&0.27&0.24&0.21&0.49&0.13&0.23\\-0.09&-0.17&-0.3&0.12&0.06&0.22&0.13&0.46&0.13\\-0.12&-0.18&-0.3&0.14&0.11&0.13&0.23&0.13&0.47\end{bmatrix} \times \begin{bmatrix}1810\\1910\\1920\\930\\880\\965\\945\\970\\950\end{bmatrix}
\end{equation*}
\begin{equation*}
\begin{bmatrix}\beta_{1}\\\beta_{2}\\\beta_{3}\\u_{1}\\u_{2}\\u_{3}\\u_{4}\\u_{5}\\u_{6}\end{bmatrix} = \begin{bmatrix}906.01\\952.8\\958.9\\8.67\\-10.68\\7.47\\-3.07\\5.93\\-3.72\end{bmatrix}
\end{equation*}
\end{enumerate}
\end{enumerate}
\end{document}
