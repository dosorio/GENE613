\documentclass[12pt,a4paper]{paper}
\usepackage[utf8]{inputenc}
\usepackage[spanish]{babel}
\usepackage{amsmath}
\usepackage{tikz}
\usepackage{pgfplots}
\usepackage[makeroom]{cancel}
\usepackage{enumitem}
\usepackage{amsfonts}
\usepackage{amssymb}
\usepackage[left=2cm,right=2cm,top=2cm,bottom=2cm]{geometry}
\usepackage{Sweave}
\begin{document}
\title{GENE613 - Homework 3\\\small{Daniel Osorio - dcosorioh@tamu.edu\\Department of Veterinary Integrative Biosciences\\Texas A\&M University}}
\maketitle
\Sconcordance{concordance:HW3_DanielOsorio.tex:HW3_DanielOsorio.Rnw:%
1 11 1 1 0 8 1 1 6 5 0 1 1 7 0 1 2 1 5 4 0 1 3 2 0 1 3 7 0 1 3 8 0 1 2 %
29 1 1 2 1 0 6 1 6 0 1 2 1 10 9 0 1 4 3 0 1 1 13 0 1 10 9 0 1 4 3 0 1 1 %
14 0 1 2 5 1}

Single locus with 2 alleles in 2 population in Hardy–Weinberg equilibrium. Given $f(A) = p_{1} = 0.3$ in population 1; and  $f(A) = p_{2} = 0.4$ in population 2; $a = -2$; $d = 1.5$.
\begin{enumerate}
\item Assume a group of migrants $m = 0.12$ from population 2 to population 1.
\begin{enumerate}
\item After this migration event, what will be the allele frequencies in each population?
\begin{Schunk}
\begin{Sinput}
> newAlleleFrequencies <- function(p1, p2, m) {
+   A = ((1 - m) * p1) + (m * p2)
+   a = 1 - A
+   return(c(A = A, a = a))
+ }
> newAlleleFrequencies(p1 = 0.3, p2 = 0.4, m = 0.12)
\end{Sinput}
\begin{Soutput}
    A     a 
0.312 0.688 
\end{Soutput}
\end{Schunk}
\item Assume ranom mating in both populations again. What will be the change in means in each population from before to current?
\begin{Schunk}
\begin{Sinput}
> populationMean <- function(p, a, d) {
+   q <- (1 - p)
+   return(p^2*a + 2*p*q*d + q^2*-a)
+ }
> meansChange <- function(Mnew, M){
+   Mnew - M
+ }
> # Population 1
> meansChange(Mnew = populationMean(p = 0.312, a = -2, d = 1.5), 
+             M = populationMean(p = 0.3, a = -2, d = 1.5))
\end{Sinput}
\begin{Soutput}
[1] -0.034032
\end{Soutput}
\begin{Sinput}
> # Population 2
> meansChange(Mnew = populationMean(p = 0.4, a = -2, d = 1.5), 
+             M = populationMean(p = 0.4, a = -2, d = 1.5))
\end{Sinput}
\begin{Soutput}
[1] 0
\end{Soutput}
\end{Schunk}
\item Put $\Delta = p_{2} - p_{1}$, originally (in terms of a single parameter representing allele frequency) $f(A) = p_{1}$ and $f(a) = 1 - p_{1}$ and now $f(A)_{new} = p_{1} + m\Delta$, and $f(a)_{new} = 1 - p_{1} - m\Delta$. The original genotypic mean was: $M = (2p_{1}-1)a + 2p_{1}(1-p_{1})d$, and after migration and return to random mating $M_{new}=(2p_{1} -1 + 2m\Delta)a + 2[p_{1}(1-p_{1})-m\Delta(2p_{1} - 1 + m\Delta)]d$. Show algebraically that the change on population mean is $M_{new} - M = 2m\Delta[a-(2p_{1}-1+m\Delta)d]$ and confirm that this is correct whit allele frequencies as above.
\begin{equation}
\begin{split}
M_{new} & = (2p_{1} -1 + 2m\Delta)a + 2[p_{1}(1-p_{1})-m\Delta(2p_{1} - 1 + m\Delta)]d \\
& = 2p_{1}a - a + 2m\Delta a + 2[p_{1}(1-p_{1})-m\Delta(2p_{1} - 1 + m\Delta)]d\\
& = 2p_{1}a - a + 2m\Delta a + 2[p_{1} - p_{1}^{2} - m\Delta(2p_{1} - 1 + m\Delta)]d \\
& = 2p_{1}a - a + 2m\Delta a + 2[p_{1} - p_{1}^{2} - 2p_{1}m\Delta - m\Delta + m^{2}\Delta^{2}]d\\
& = 2p_{1}a - a + 2m\Delta a + 2dp_{1} - 2dp_{1}^{2} - 4dp_{1}m\Delta - 2dm\Delta + 2dm^{2}\Delta^{2}\\
\end{split}
\end{equation}
\begin{equation}
\begin{split}
M & = (2p_{1}-1)a + 2p_{1}(1-p_{1})d \\
& = 2p_{1}a - a + 2p_{1}(1-p_{1})d \\
& = 2p_{1}a - a + 2dp_{1}- 2dp_{1}^{2} \\
\end{split}
\end{equation}
\begin{equation}
\begin{split}
2m\Delta[a-(2p_{1}-1+m\Delta)d] & = 2p_{1}a - a + 2m\Delta a + 2dp_{1} - 2dp_{1}^{2} - 4dp_{1}m\Delta\\& - 2dm\Delta + 2dm^{2}\Delta^{2} - [2p_{1}a - a + 2dp_{1}- 2dp_{1}^{2}] \\
2m\Delta[a-(2p_{1}-1+m\Delta)d] & = 2p_{1}a - a + 2m\Delta a + 2dp_{1} - 2dp_{1}^{2} - 4dp_{1}m\Delta\\& - 2dm\Delta + 2dm^{2}\Delta^{2} - 2p_{1}a + a - 2dp_{1}+ 2dp_{1}^{2} \\
2m\Delta[a-(2p_{1}-1+m\Delta)d] & = \cancel{2p_{1}a} - a + 2m\Delta a + 2dp_{1} - 2dp_{1}^{2} - 4dp_{1}m\Delta\\& - 2dm\Delta + 2dm^{2}\Delta^{2} \cancel{- 2p_{1}a} + a - 2dp_{1}+ 2dp_{1}^{2} \\
2m\Delta[a-(2p_{1}-1+m\Delta)d] & = \cancel{- a} + 2m\Delta a + 2dp_{1} - 2dp_{1}^{2} - 4dp_{1}m\Delta\\& - 2dm\Delta + 2dm^{2}\Delta^{2} \cancel{+ a} - 2dp_{1}+ 2dp_{1}^{2} \\
2m\Delta[a-(2p_{1}-1+m\Delta)d] & = 2m\Delta a \cancel{+ 2dp_{1}} - 2dp_{1}^{2} - 4dp_{1}m\Delta\\& - 2dm\Delta + 2dm^{2}\Delta^{2} \cancel{- 2dp_{1}} + 2dp_{1}^{2} \\
2m\Delta[a-(2p_{1} - 1 + m\Delta)d] & = 2m\Delta a \cancel{- 2dp_{1}^{2}} - 4dp_{1}m\Delta - 2dm\Delta + 2dm^{2}\Delta^{2} \cancel{ + 2dp_{1}^{2}} \\
2m\Delta[a-(2p_{1}-1+m\Delta)d] & = 2m\Delta a - 4dp_{1}m\Delta - 2dm\Delta + 2dm^{2}\Delta^{2} \\
2m\Delta[a-(2p_{1}-1+m\Delta)d] & = 2m\Delta[a - 2dp_{1} - d + dm\Delta]\\
2m\Delta[a-(2p_{1}-1+m\Delta)d] & = 2m\Delta[a - (2p_{1} - 1 + m\Delta)d]\\
\end{split}
\end{equation}
\begin{Schunk}
\begin{Sinput}
> p1 <- 0.3
> p2 <- 0.4
> m <- 0.12
> a = -2
> d = 1.5
> delta <- (p2 - p1)
> 2 * m * delta * (a - (2 * p1 - 1 +  (m * delta)) * d)
\end{Sinput}
\begin{Soutput}
[1] -0.034032
\end{Soutput}
\end{Schunk}
\item Use migration proportions from 0.05 to 0.45 in increments of 0.05 and allele frequency differences between populations ($\Delta$) from 0 to 0.3 in increments of 0.05, develop a matrix of allele frequencies in the receiving population and a matrix of changes in mean of the receiving population after return to random mating. Make both matrices with respect to the initial frequency of the $A$ allele in the receiving population, that is, do not make them cumulative. Briefly interpret the results presented in each.
\begin{Schunk}
\begin{Sinput}
> alleleF <- sapply(seq(0.05, 0.45, 0.05), function(mProportion) {
+   sapply(seq(0, 0.3, 0.05), function(alleleDiff) {
+     round(newAlleleFrequencies(
+       p1 = 0.3,
+       p2 = (0.3 + alleleDiff),
+       m = mProportion
+     )[1], 2)
+   })
+ })
> dimnames(alleleF) <- list(
+   Delta = seq(0, 0.3, 0.05),
+   migrationProportion = seq(0.05, 0.45, 0.05)
+ )
> alleleF
\end{Sinput}
\begin{Soutput}
      migrationProportion
Delta  0.05  0.1 0.15  0.2 0.25  0.3 0.35  0.4 0.45
  0    0.30 0.30 0.30 0.30 0.30 0.30 0.30 0.30 0.30
  0.05 0.30 0.30 0.31 0.31 0.31 0.32 0.32 0.32 0.32
  0.1  0.30 0.31 0.32 0.32 0.32 0.33 0.34 0.34 0.34
  0.15 0.31 0.32 0.32 0.33 0.34 0.34 0.35 0.36 0.37
  0.2  0.31 0.32 0.33 0.34 0.35 0.36 0.37 0.38 0.39
  0.25 0.31 0.32 0.34 0.35 0.36 0.38 0.39 0.40 0.41
  0.3  0.31 0.33 0.34 0.36 0.38 0.39 0.40 0.42 0.44
\end{Soutput}
\begin{Sinput}
> meanChanges <- sapply(seq(0.05, 0.45, 0.05), function(mProportion) {
+   sapply(seq(0, 0.3, 0.05), function(alleleDiff) {
+     originalMean <- populationMean(p = 0.3, a = -2, d = 1.5)
+     newAlleleF <- newAlleleFrequencies(p1 = 0.3,
+                                        p2 = (0.3 + alleleDiff),
+                                        m = mProportion)[1]
+     newMean <- populationMean(p = newAlleleF, a = -2, d = 1.5)
+     round(meansChange(Mnew = newMean, M = originalMean), 2)
+   })
+ })
> dimnames(meanChanges) <- list(
+   Delta = seq(0, 0.3, 0.05),
+   migrationProportion = seq(0.05, 0.45, 0.05)
+ )
> meanChanges
\end{Sinput}
\begin{Soutput}
      migrationProportion
Delta   0.05   0.1  0.15   0.2  0.25   0.3  0.35   0.4  0.45
  0     0.00  0.00  0.00  0.00  0.00  0.00  0.00  0.00  0.00
  0.05 -0.01 -0.01 -0.02 -0.03 -0.04 -0.04 -0.05 -0.06 -0.06
  0.1  -0.01 -0.03 -0.04 -0.06 -0.07 -0.09 -0.10 -0.12 -0.13
  0.15 -0.02 -0.04 -0.06 -0.09 -0.11 -0.13 -0.16 -0.18 -0.20
  0.2  -0.03 -0.06 -0.09 -0.12 -0.15 -0.18 -0.21 -0.24 -0.28
  0.25 -0.04 -0.07 -0.11 -0.15 -0.19 -0.23 -0.27 -0.31 -0.35
  0.3  -0.04 -0.09 -0.13 -0.18 -0.23 -0.28 -0.33 -0.38 -0.43
\end{Soutput}
\end{Schunk}
\begin{description}[align=left]
\item [Interpretation:] The first matrix shows how the frequency of the allele A $f(A)$ change gradually in function of the changes of  the $\Delta$ parameter and the $m$ proportion. As we can see, the $f(A)$ is more affected in two scenarios, when the $\Delta$ is high and the proportion of migrants is also high. The same effect is shown in the second matrix, where the differences in the population mean are shown after a migration event.
\end{description}
\end{enumerate}
\end{enumerate}
\end{document}
